%%% Local Variables:
%%% coding: utf-8
%%% mode: latex
%%% TeX-engine: xetex
%%% TeX-master: t
%%% End:

\documentclass[6pt]{article}
\usepackage[utf8]{inputenc}
\usepackage{tikz}
\usetikzlibrary{backgrounds}
\usepackage{fp}
\usepackage{forloop}
% \usepackage{fontspec}
\usepackage{pgfkeys}
% \usepackage{amsmath}
% \usepackage{mathtools}
\usepackage {accents}
\usepackage{mathspec}
\usepackage {bm}

\usepackage[a4paper, total={180mm, 277mm}]{geometry}

\setmainfont[
Path = /home/tarn/repo/code/py/pinouts/,
UprightFont = FuturaPTBold,
Extension = .otf,
SizeFeatures={Size=7}
]{FuturaPTBold}

\setmathsfont(Digits,Latin)[
Path = /home/tarn/repo/code/py/pinouts/,
UprightFont = FuturaPTBold,
ItalicFont = FuturaPTBold,
BoldFont = FuturaPTBold,
BoldItalicFont = FuturaPTBold,
Extension = .otf,
Uppercase=Regular,
Lowercase=Regular,
Arabic=Regular,
SizeFeatures={Size=7}
]{FuturaPTBold}

\begin{document}


%%%%%%%%%%%%%%%%%%%%%%%%%%%%%%%%%%%%%%%%%%%%%%%%%%%%%%%%%%%%%%%%%%%%%%%%%%%%%%%%
% MYSTERY
%%%%%%%%%%%%%%%%%%%%%%%%%%%%%%%%%%%%%%%%%%%%%%%%%%%%%%%%%%%%%%%%%%%%%%%%%%%%%%%%
\pgfkeys{
 /pinnames/.is family, /pinnames,
 .unknown/.style = {\pgfkeyscurrentname/.initial = #1},
}
\newcommand\setpinnames[1]{\pgfkeys{/pinnames, #1}}
\newcommand\getpinnames[1]{\pgfkeysvalueof{/pinnames/#1}}
%%%%%%%%%%%%%%%%%%%%%%%%%%%%%%%%%%%%%%%%%%%%%%%%%%%%%%%%%%%%%%%%%%%%%%%%%%%%%%%%
% /MYSTERY
%%%%%%%%%%%%%%%%%%%%%%%%%%%%%%%%%%%%%%%%%%%%%%%%%%%%%%%%%%%%%%%%%%%%%%%%%%%%%%%%

\newcommand{\ic}[4]{
% ??
% \wlog {#4}
% \newcounter{ct}
% \foreach \x in {#4} {
%   \stepcounter{ct}
%   \setpinnames{\thect=\x}
% \wlog{\thect}
% }
\setpinnames{#4}
% \begin{tikzpicture}[framed,line cap=round]
\begin{tikzpicture}[line cap=round]
\FPeval\aspect{1.5}
\FPeval\npins{#1}
\FPeval\halfpins{#1 / 2}
\FPeval\width{20.0}
% \FPeval\height{\npins * \aspect * \width / 16}
\FPeval\pinpitch{11.0 * \width / 66.0}
\FPeval\height{(\halfpins + 1) * \pinpitch}

\FPeval\xpos{0}
\FPeval\ypos{0}

\FPeval\boxx{\xpos + \pinpitch}
\FPeval\boxy{\ypos}

% \FPeval\dbga{\ypos}
% \FPeval\dbgb{\boxy}
% \draw[line width=0.5mm] (0 mm, \ypos mm) -- (40 mm, \ypos mm);
% \draw[line width=0.5mm] (0 mm, \boxy mm) -- (40 mm, \boxy mm);

\FPeval\xa{\boxx}
\FPeval\ya{\boxy}
\FPeval\xb{\boxx + \width}
\FPeval\yb{\boxy + \height}

% Left, right, and bottom sides of the box.
\draw[line width=0.5mm] (\xa mm,\ya mm) -- (\xb mm, \ya mm);
\draw[line width=0.5mm] (\xa mm,\ya mm) -- (\xa mm, \yb mm);
\draw[line width=0.5mm] (\xb mm,\ya mm) -- (\xb mm, \yb mm);

\useasboundingbox (\xa mm, \ya mm) rectangle (\xb mm, \yb mm);
% \draw[use as bounding box] (\xa mm, \ya mm) rectangle (\xb mm, \yb mm);

% Now make the dimple.
\FPeval\xcp{\xa + \width / 2}
\FPeval\xlp{\xcp - \pinpitch / 2}
\FPeval\xrp{\xcp + \pinpitch / 2}
\draw[line width=0.5mm] (\xa mm,\yb mm) -- (\xlp mm, \yb mm);
\draw[line width=0.5mm] (\xrp mm,\yb mm) -- (\xb mm, \yb mm);
\FPeval\dia{\pinpitch / 2}
\begin{scope}
    \clip (\xa mm, \ya mm) rectangle (\xb mm, \yb mm);
    \draw[line width=0.5mm] (\xcp mm, \yb mm) circle(\dia mm);
\end{scope}

% Add name and caption
\FPeval\namey{\ya - \pinpitch / 4}
\FPeval\descy{\ya - \pinpitch * 1.25}
% \FPeval\namey{\yb + \pinpitch * 3}
% \FPeval\descy{\yb + \pinpitch * 2}
\node[anchor=north,align=center] at (\xcp mm, \namey mm) {#2};
\node[anchor=north,align=center] at (\xcp mm, \descy mm) {#3};

% Pin 1 marker
\FPeval\pxa{\xpos}
\FPeval\pxb{\pxa + \pinpitch}
\FPeval\pinonemarkerx{\pxa + \pinpitch * 2}
\FPeval\pinonemarkery{\ypos + \pinpitch * \halfpins}
\FPeval\dotsize{\pinpitch / 4}
\draw[black,fill=black] (\pinonemarkerx mm,\pinonemarkery mm) circle (\dotsize mm);

% Left side pins
\foreach \ct in {1,...,\halfpins}
{%
\FPeval\pya{\ypos + \ct * \pinpitch}
\FPeval\pyb{\pya}
\draw[line width=0.5mm] (\pxa mm, \pya mm) -- (\pxb mm, \pyb mm);

% Pin number
\FPeval\pnx{\pxb}
\FPeval\pny{\pyb}
\FPeval\pn{round(\halfpins + 1 - \ct,0)}
\node[anchor=west,align=left] at (\pnx mm,\pny mm) {\pn};

% Pin label
\FPeval\plx{\pxa}
\FPeval\ply{\pyb}
\node[anchor=east] at (\plx mm,\ply mm) {\getpinnames{\pn}};
}

% Right side pins
\FPeval\pxa{\boxx + \width}
\FPeval\pxb{\pxa + \pinpitch}
\foreach \ct in {1,...,\halfpins}
{%
\FPeval\pya{\ypos + \ct * \pinpitch}
\FPeval\pyb{\pya}
\draw[line width=0.5mm] (\pxa mm, \pya mm) -- (\pxb mm, \pyb mm);

% Pin number
\FPeval\pnx{\pxa}
\FPeval\pny{\pyb}
\FPeval\pn{round(\halfpins + \ct,0)}
\node[anchor=east,align=right] at (\pnx mm,\pny mm) {\pn};

% Pin label
\FPeval\plx{\pxb}
\FPeval\ply{\pyb}
\node[anchor=west] at (\plx mm,\ply mm) {\getpinnames{\pn}};
}
\end{tikzpicture}}

% \newcommand*\inv[1]{\overbracket[0.5mm][-1mm]{#1}}
% \newcommand*\inv[1]{\bar{#1}}
\newcommand*\inv[1]{\overline{#1}}
% \newcommand\inv[1]{\accentset{\rule{0.8em}{1pt}}{#1}}
% \newcommand\inv[1]{\neg{#1}}

\newcommand\VDD{$V_{DD}$}
\newcommand\VSS{$V_{SS}$}

\begin{center}
\bgroup
\setlength\tabcolsep{20mm}
% \setlength\tabrowsep{25mm}
\def\arraystretch{10}%
\begin{tabular}{ccc}

\ic {14} {4011} {Quad NAND}
  {1=A, 2=B, 3=$\inv{AB}$, 4=$\inv{CD}$, 5=C, 6=D, 7=\VSS, 8=E, 9=F, 10=$\inv{EF}$, 11=$\inv{GH}$, 12=G, 13=H, 14=\VDD}
&
\ic {14} {4013} {Dual D-Type Flip-Flop}
  {1=$Q1$, 2=$\inv{Q1}$, 3=CLOCK1, 4=RESET1, 5=D1, 6=SET1, 7=\VSS, 8=SET2, 9=D2, 10=RESET2, 11=CLOCK2, 12=$\inv{Q2}$, 13=Q2, 14=\VDD, 15=$\inv F$, 16=NC}
&
\ic {16} {4017} {Decade Counter}
  {1=5, 2=1, 3=0, 4=2, 5=6, 6=7, 7=3, 8=\VSS, 16=\VDD, 15=RESET, 14=CLOCK, 13=CLOCK INHIBIT, 12=CARRY OUT, 11=9, 10=4, 9=8}
\\
\ic {16} {4042} {Quad Clocked D Latch}
  {1=Q4, 2=Q1, 3=$\inv{Q1}$, 4=D1, 5=CLOCK, 6=POLARITY, 7=D2, 8=\VSS, 9=$\inv{Q2}$, 10=Q2, 11=Q3, 12=$\inv{Q3}$, 13=D3, 14=D4, 15=$\inv{Q4}$, 16=\VDD}
&
\ic {16} {4043} {Quad 3-State R/S Latch}
  {1=Q4, 2=Q1, 3=R1, 4=S1, 5=ENABLE, 6=S2, 7=R2, 8=\VSS, 9=Q2, 10=Q3, 11=R3, 12=S3, 13=NC, 14=S4, 15=R4, 16=\VDD}
&
\ic {16} {4049} {Hex Inverter}
  {1=\VDD, 3=A, 2=$\inv A$, 5=B, 4=$\inv B$, 7=C, 6=$\inv C$, 8=\VSS, 9=D, 10=$\inv D$, 11=E, 12=$\inv E$, 13=NC, 14=F, 15=$\inv F$, 16=NC}
\\
\ic {16} {4060} {Oscillator/Counter}
  {1=Q12, 2=Q13, 3=Q14, 4=Q6, 5=Q5, 6=Q7, 7=Q4, 8=\VSS, 9=${\bf\phi}_O$, 10=$\inv{{\bf\phi}_O}$, 11=${\bf\phi}_I$, 12=RESET, 13=Q9, 14=Q8, 15=Q10, 16=\VDD}
&
\ic {14} {4081} {Quad AND}
  {1=A, 2=B, 3=$AB$, 4=$CD$, 5=C, 6=D, 7=\VSS, 8=E, 9=F, 10=$EF$, 11=$GH$, 12=G, 13=H, 14=\VDD}
&
\ic {16} {4094} {SIPO Bus Register}
  {1=STROBE, 2=DATA, 3=CLOCK, 4=Q1, 5=Q2, 6=Q3, 7=Q4, 8=\VSS, 9=$Q_S$, 10=$\inv{Q_S}$, 11=Q8, 12=Q7, 13=Q6, 14=Q5, 15=OUTPUT ENABLE, 16=\VDD}
\\
\ic {14} {40106} {Hex Schmitt Trigger Inverter} {1=A, 2=$\inv A$, 3=B, 4=$\inv B$, 5=C, 6=$\inv C$, 7=\VSS, 8=$\inv D$, 9=D, 10=$\inv E$, 11=E, 12=$\inv F$, 13=F, 14=\VDD}
&
\ic {16} {670} {4x4 Register File}
  {1=D1, 2=D2, 3=D3, 4=RA1, 5=RA0, 6=Q3, 7=Q2, 8=\VSS, 9=Q1, 10=Q0, 11=$\inv{RE}$, 12=$\inv{WE}$, 13=WA1, 14=WA0, 15=D0, 16=\VDD}
&
\ic {16} {283} {4-bit Adder}
{1=S1, 2=B1, 3=A1, 4=S0, 5=A0, 6=B0, 7=$C_{IN}$, 8=\VSS, 9=$C_{OUT}$, 10=S3, 11=B3, 12=A3, 13=S2, 14=A2, 15=B2, 16=\VDD}
\\
\ic {28} {PIC18F27K42} {}
{1=Vpp/$\inv{MCLR}$/RE3, 2=RA0, 3=RA1, 4=RA2, 5=RA3, 6=RA4, 7=RA5, 8=\VSS, 9=RA7, 10=RA6, 11=RC0, 12=RC1, 13=RC2, 14=RC3, 15=RC4, 16=RC5, 17=RC6, 18=RC7, 19=\VSS, 20=\VDD, 21=RB0, 22=RB1, 23=RB2, 24=RB3, 25=RB4, 26=RB5, 27=RB6/ICSPCLK, 28=RB7/ICSPDAT}
&
\ic {28} {DS1230Y} {Real-Time Clock}
{1=A14, 2=A12, 3=A7, 4=A6, 5=A5, 6=A4, 7=A3, 8=A2, 9=A1, 10=A0, 11=DQ0, 12=DQ1, 13=DQ2, 14=\VSS, 15=DQ3, 16=DQ4, 17=DQ5, 18=DQ6, 19=DQ7, 20=$\inv{CHIP\ ENABLE}$, 21=A10, 22=$\inv{OUTPUT\ ENABLE}$, 23=A11, 24=A9, 25=A8, 26=A13, 27=$\inv{WRITE\ ENABLE}$, 28=\VDD}
&
\ic {32} {SST39SF010A} {1 Mbit Flash Memory}
{1=NC, 2=A16, 3=A15, 4=A12, 5=A7, 6=A6, 7=A5, 8=A4, 9=A3, 10=A2, 11=A1, 12=A0, 13=DQ0, 14=DQ1, 15=DQ2, 16=\VSS, 17=DQ3, 18=DQ4, 19=DQ5, 20=DQ6, 21=DQ7, 22=$\inv{CHIP\ ENABLE}$, 23=A10, 24=$\inv{OUTPUT\ ENABLE}$, 25=A11, 26=A9, 27=A8, 28=A13, 29=A14, 30=NC, 31=$\inv{WRITE\ ENABLE}$, 32=\VDD}
\\
\end{tabular}

\begin{tabular}{ccc}
\ic {16} {4015} {Dual 4-bit Shift Register}
{1=B CLOCK, 2=BQ4, 3=AQ3, 4=AQ2, 5=AQ1, 6=A RESET, 7=A DATA, 8=\VSS, 9=A CLOCK, 10=AQ4, 11=BQ3, 12=BQ2, 13=BQ1, 14=B RESET, 15=B DATA, 16=\VDD}
&
\ic {16} {4028} {BCD to 1-of-10 Decoder}
{1=4, 2=2, 3=0, 4=7, 5=9, 6=5, 7=6, 8=\VSS, 9=8, 10=A, 11=D, 12=C, 13=B, 14=1, 15=3, 16=\VDD}
&
\ic {16} {4031} {64-bit shift register}
{1=DATA IN 2, 2=CLOCK IN, 3=NC, 4=NC, 5=$Q'$, 6=Q, 7=$\inv{Q}$, 8=\VSS, 9=DELAYED CLOCK OUT, 10=MODE CONTROL, 11=NC, 12=NC, 13=NC, 14=NC, 15=DATA IN 1, 16=\VDD}
\\
\ic {16} {4035} {4-bit Parallel I/O Shift Register}
{1=Q1, 2=TRUE/COMP, 3=$\inv{K}$, 4=J, 5=RESET, 6=CLOCK, 7=P/S, 8=\VSS, 9=PI-1, 10=PI-2, 11=PI-3, 12=PI-4, 13=Q4, 14=Q3, 15=Q2, 16=\VDD}
&
\ic {16} {4046} {PLL}
{1=PHASE PULSES, 2=PHASE COMP I, 3=COMPARATOR IN, 4=VCO OUT, 5=INHIBIT, 6=CI1, 7=CI2, 8=\VSS, 9=VCO IN, 10=DEMODULATOR OUT, 11=R1 TO \VSS, 12=R2 TO \VSS, 13=PHASE COMP II OUT, 14=SIGNAL IN, 15=ZENER, 16=\VDD}
&
\ic {16} {4050} {Hex Buffer}
  {1=\VDD, 3=I1, 2=O1, 5=I2, 4=O2, 7=I3, 6=O3, 8=\VSS, 9=I4, 10=O4, 11=I5, 12=O5, 13=NC, 14=I6, 15=O6, 16=NC}
\\
\ic {16} {4536} {Programmable Timer}
{1=SET, 2=RESET, 3=IN 1, 4=OUT 1, 5=OUT 2, 6=8-BYPASS, 7=CLOCK INHIBIT, 8=\VSS, 9=A, 10=B, 11=C, 12=D, 13=DECODE OUT, 14=OSC INHIBIT, 15=MONO IN, 16=\VDD}
\\
\end{tabular}
\egroup
\end{center}

\end{document}
%%% Local Variables:
%%% mode: latex
%%% TeX-master: t
%%% End:
