%%% Local Variables:
%%% coding: utf-8
%%% mode: latex
%%% TeX-engine: xetex
%%% TeX-master: t
%%% End:

\documentclass[6pt]{article}
\usepackage[utf8]{inputenc}
\usepackage{tikz}
\usepackage{fp}
\usepackage{forloop}
% \usepackage{fontspec}
\usepackage{pgfkeys}
% \usepackage{amsmath}
% \usepackage{mathtools}
\usepackage {accents}
\usepackage{mathspec}
\usepackage {bm}

\setmainfont[
Path = /home/tarn/repo/code/py/pinouts/,
UprightFont = FuturaPTBold,
Extension = .otf,
SizeFeatures={Size=7}
]{FuturaPTBold}

\setmathsfont(Digits,Latin)[
Path = /home/tarn/repo/code/py/pinouts/,
UprightFont = FuturaPTBold,
ItalicFont = FuturaPTBold,
BoldFont = FuturaPTBold,
BoldItalicFont = FuturaPTBold,
Extension = .otf,
Uppercase=Regular,
Lowercase=Regular,
Arabic=Regular
]{FuturaPTBold}

\begin{document}


%%%%%%%%%%%%%%%%%%%%%%%%%%%%%%%%%%%%%%%%%%%%%%%%%%%%%%%%%%%%%%%%%%%%%%%%%%%%%%%%
% MYSTERY
%%%%%%%%%%%%%%%%%%%%%%%%%%%%%%%%%%%%%%%%%%%%%%%%%%%%%%%%%%%%%%%%%%%%%%%%%%%%%%%%
\pgfkeys{
 /pinnames/.is family, /pinnames,
 .unknown/.style = {\pgfkeyscurrentname/.initial = #1},
}
\newcommand\setpinnames[1]{\pgfkeys{/pinnames, #1}}
\newcommand\getpinnames[1]{\pgfkeysvalueof{/pinnames/#1}}
%%%%%%%%%%%%%%%%%%%%%%%%%%%%%%%%%%%%%%%%%%%%%%%%%%%%%%%%%%%%%%%%%%%%%%%%%%%%%%%%
% /MYSTERY
%%%%%%%%%%%%%%%%%%%%%%%%%%%%%%%%%%%%%%%%%%%%%%%%%%%%%%%%%%%%%%%%%%%%%%%%%%%%%%%%

\newcommand{\ic}[4]{

% ??
% \wlog {#4}
% \newcounter{ct}
% \foreach \x in {#4} {
%   \stepcounter{ct}
%   \setpinnames{\thect=\x}
% \wlog{\thect}
% }

\setpinnames{#4}

\begin{tikzpicture}[line cap=round]

\FPeval\aspect{1.5}
\FPeval\npins{#1}
\FPeval\halfpins{#1 / 2}
\FPeval\width{20.0}
% \FPeval\height{\npins * \aspect * \width / 16}
\FPeval\pinpitch{11.0 * \width / 66.0}
\FPeval\height{(\halfpins + 1) * \pinpitch}

\FPeval\xpos{0}
\FPeval\ypos{0}

\FPeval\boxx{\xpos + \pinpitch}
\FPeval\boxy{\ypos}

% \FPeval\dbga{\ypos}
% \FPeval\dbgb{\boxy}
% \draw[line width=0.5mm] (0 mm, \ypos mm) -- (40 mm, \ypos mm);
% \draw[line width=0.5mm] (0 mm, \boxy mm) -- (40 mm, \boxy mm);

\FPeval\xa{\boxx}
\FPeval\ya{\boxy}
\FPeval\xb{\boxx + \width}
\FPeval\yb{\boxy + \height}

% Left, right, and bottom sides of the box.
\draw[line width=0.5mm] (\xa mm,\ya mm) -- (\xb mm, \ya mm);
\draw[line width=0.5mm] (\xa mm,\ya mm) -- (\xa mm, \yb mm);
\draw[line width=0.5mm] (\xb mm,\ya mm) -- (\xb mm, \yb mm);

% Now make the dimple.
\FPeval\xcp{\xa + \width / 2}
\FPeval\xlp{\xcp - \pinpitch / 2}
\FPeval\xrp{\xcp + \pinpitch / 2}
\draw[line width=0.5mm] (\xa mm,\yb mm) -- (\xlp mm, \yb mm);
\draw[line width=0.5mm] (\xrp mm,\yb mm) -- (\xb mm, \yb mm);
\FPeval\dia{\pinpitch / 2}
\begin{scope}
    \clip (\xa mm, \ya mm) rectangle (\xb mm, \yb mm);
    \draw[line width=0.5mm] (\xcp mm, \yb mm) circle(\dia mm);
\end{scope}

% \FPeval\namey{\ya - \pinpitch / 4}
% \FPeval\descy{\ya - \pinpitch * 1.25}
\FPeval\namey{\yb + \pinpitch * 3}
\FPeval\descy{\yb + \pinpitch * 2}
\node[anchor=north,align=center] at (\xcp mm, \namey mm) {#2};
\node[anchor=north,align=center] at (\xcp mm, \descy mm) {#3};



\FPeval\pxa{\xpos}
\FPeval\pxb{\pxa + \pinpitch}
\FPeval\pinonemarkerx{\pxa + \pinpitch * 2}
\FPeval\pinonemarkery{\ypos + \pinpitch * \halfpins}
\FPeval\dotsize{\pinpitch / 4}
\draw[black,fill=black] (\pinonemarkerx mm,\pinonemarkery mm) circle (\dotsize mm);

\foreach \ct in {1,...,\halfpins}
{%
\FPeval\pya{\ypos + \ct * \pinpitch}
\FPeval\pyb{\pya}
\draw[line width=0.5mm] (\pxa mm, \pya mm) -- (\pxb mm, \pyb mm);

\FPeval\pnx{\pxb}
\FPeval\pny{\pyb}
\FPeval\pn{round(\halfpins + 1 - \ct,0)}
\node[anchor=west,align=left] at (\pnx mm,\pny mm) {\pn};

\FPeval\plx{\pxa}
\FPeval\ply{\pyb}
\node[anchor=east] at (\plx mm,\ply mm) {\getpinnames{\pn}};
}

\FPeval\pxa{\boxx + \width}
\FPeval\pxb{\pxa + \pinpitch}
\foreach \ct in {1,...,\halfpins}
{%
\FPeval\pya{\ypos + \ct * \pinpitch}
\FPeval\pyb{\pya}
\draw[line width=0.5mm] (\pxa mm, \pya mm) -- (\pxb mm, \pyb mm);

\FPeval\pnx{\pxa}
\FPeval\pny{\pyb}
\FPeval\pn{round(\halfpins + \ct,0)}
\node[anchor=east,align=right] at (\pnx mm,\pny mm) {\pn};

\FPeval\plx{\pxb}
\FPeval\ply{\pyb}
\node[anchor=west] at (\plx mm,\ply mm) {\getpinnames{\pn}};
}
\end{tikzpicture}

}

% \newcommand*\inv[1]{\overbracket[0.5mm][-1mm]{#1}}
% \newcommand*\inv[1]{\bar{#1}}
\newcommand\inv[1]{\accentset{\rule{0.8em}{1pt}}{#1}}
% \newcommand\inv[1]{\neg{#1}}

% \ic {8}
% \ic {10}
% \ic {12}
% \ic {14}
% \setpinnames
% 15=RESET, 14=CLOCK, 13=CLOCK INHIBIT, 12=CARRY OUT, 11=9, 10=4, 9=8, $V_{dd}$}
\ic {16} {4017} {Decade Counter} {1=5, 2=1, 3=0, 4=2, 5=6, 6=7, 7=3, 8=$V_{SS}$, 16=$V_{DD}$, 15=RESET, 14=CLOCK, 13=CLOCK INHIBIT, 12=CARRY OUT, 11=9, 10=4, 9=8}
% \ic {16} {4049} {Hex Inverter} {1=$V_{CC}$, 2=A, 3=$\inv A$, 4=B, 5=$\inv B$, 6=C, 7=$\inv C$, 8=$V_{SS}$, 9=D, 10=$\inv D$, 11=E, 12=$\inv E$, 13=NC, 14=F, 15=$\inv F$, 16=NC}
% \ic {14} {4013} {Dual D-Type Flip-Flop} {1=$Q1$, 2=$\inv{Q1}$, 3=CLOCK1, 4=RESET1, 5=D1, 6=SET1, 7=$V_{SS}$, 8=SET2, 9=D2, 10=RESET2, 11=CLOCK2, 12=$\inv{Q2}$, 13=Q2, 14=$V_{DD}$, 15=$\inv F$, 16=NC}
% \ic {18}
% \ic {20}
% \ic {22}
% \ic {24}
% \ic {26}
% \ic {28}

\end{document}